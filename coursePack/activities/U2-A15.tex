<<<<<<< HEAD
\def\theTopic{Arsenic}
\def\dayNum{15 }

\begin{center}
{\bf {\large Arsenic in Toenails}}
\end{center}

 Symptoms of low--level  arsenic poisoning
include headaches, confusion, severe diarrhea and
drowsiness.  When the poisoning becomes acute, symptoms include
vomiting, blood in the urine, hair loss, convulsions, and even death.
A 2007 study by Peter Ravenscroft found that over 137 million people
in more than 70 countries are probably affected by arsenic poisoning
from drinking water.\footnote{Ravenscroft, P. (2007). The global
  dimensions of arsenic pollution of groundwater. {\em Tropical
    Agriculture Association}, {\bf 3}.}

Scientists can assay toe nail clippings to measure a person's arsenic
level in parts per million (ppm).  They did this assay on 19 randomly selected 
individuals who drink from private wells in New Hampshire (data in the
table below). They want to know  the mean arsenic 
concentration for New Hampshire residents drinking from private wells.

 An arsenic level greater than 0.150 ppm is considered
hazardous. A secondary question is, `` Is there evidence that people
drinking the ground water in New Hampshire are suffering from arsenic
poisoning?''


\begin{center}
\begin{tabular}{|r|r|r|r|r|r|r|r|r|r|} \hline
0.119& 0.118& 0.099& 0.118& 0.275& 0.358& 0.080& 0.158& 0.310& 0.105
\\ \hline
0.073& 0.832& 0.517& 0.851& 0.269& 0.433& 0.141& 0.135& 0.175 & \\ \hline
\end{tabular}  
\end{center}


{\bf Step 1. State the research question. }
\begin{enumerate}
\item  \label{researchQs}Based on the description of the study, state the two research
  questions to be answered.
=======
\def\theTopic{Errors }
\def\dayNum{15 }

\begin{center}
\vspace*{-.2in}
{\bf {\large On Being Wrong 5\% of the Time}}\vspace{-.6cm}
\end{center}


Our confidence in a 95\% confidence interval comes from the fact that,
in the long run, the technique works 95\% of the time to capture the
unknown parameter.  This leads to an old cheap joke:

{\sf Statisticians are people who require themselves to be wrong 5\%
  of the time.}

We hope that's not really true, but decision making leads to a dilemma:\\
  If you want to never be wrong, you have to always put off decisions
  and collect more data.

Statistics allows us to make decisions based on partial data  while
controlling our error rates.\\
Discuss these situations and decide which error would be worse:\vspace{-.6cm}
\begin{enumerate}
\item A criminal jury will make an error if
      they let a guilty defendant go free, or if
      they convict an innocent defendant.
     Which is worse? Why?
>>>>>>> f4001662d4281dfbcc417e2cc1ec4f6d088d24a6
\begin{students}
  \vspace{1.5cm}
\end{students}

<<<<<<< HEAD
\begin{key}
  {\it 
    \begin{enumerate}
    \item How high is the mean arsenic level for New Hampshire residents
 drinking from a private well?
    \item Is there evidence that people drinking the ground water in New
 Hampshire are building up a hazardous  level (over 0.15 on average)
 of arsenic concentration? 

    \end{enumerate}
}
\end{key}
  

\item  Which research question should be answered using a hypothesis
  test and which should be answered using a confidence interval? 
=======
\item The doctor gives patients a test designed to detect pancreatic
  cancer (which is usually quite serious).  The test is wrong if:
  it says a healthy patient has cancer (a false positive), or if
  it says a patient with cancer is healthy (a false negative).  Which
  is worse?  Why? 
>>>>>>> f4001662d4281dfbcc417e2cc1ec4f6d088d24a6
\begin{students}
  \vspace{1.5cm}
\end{students}

<<<<<<< HEAD
\begin{key}
  {\it The first should be answered with a CI because we are trying to
    estimate the parameter,  and the second with a test (is there evidence?)}
\end{key}
\end{enumerate}

{\bf Step 2. Design a study and collect data. }
  \begin{enumerate}
  \setcounter{enumi}{2}
   \item  What is the variable in the study?  Is this variable quantitative
     or categorical?
=======
\item  A weather forecaster working at an airport in Indonesia on
  December 28, 2014 had to decide if it was too dangerous
   to allow Air Asia Flight 8501 to fly to
  Singapore.  The flight was allowed, resulting in the deaths of all
  162 people aboard.  Errors don't get much worse than that, but what
  would the cost be of grounding a flight?  
>>>>>>> f4001662d4281dfbcc417e2cc1ec4f6d088d24a6
\begin{students}
  \vspace{2.5cm}
\end{students}

<<<<<<< HEAD
\begin{key}
  {\it Arsenic concentration in toenails (ppm), quantitative.}
\end{key}

\item 
Define the parameter of interest in the context of the study.  What
notation should be used to denote it?
\begin{students}
  \vspace*{1cm}
\end{students}

\begin{key}
  {\it notation: $\mu$, is mean arsenic concentration in toenails for New Hampshire residents with a private well.}
\end{key}
\end{enumerate}

{\bf Step 3. Explore the data. }\\
With quantitative data, we typically report and study the average value, or the mean.
\begin{enumerate}
\setcounter{enumi}{4}
\item  What is the sample size in this study?  n = \begin{key}
  {\it 19}
\end{key}

\item \label{xbar} Calculate the observed statistic and use correct notation to
  denote it (check your answer with another  group!).
\begin{students}
  \vspace{1cm}
\end{students}
\begin{key}
   $ \xb =   0.272$. 
\end{key}
\item  Could your answer to \ref{xbar} have happened if the arsenic
  concentrations in New Hampshire residents are not hazardous? 
\begin{students}
  \vspace{1cm}
\end{students}
\begin{key}
   {\it Yes, pretty much anything is possible if at least a few people
     have high arsenic levels.}  
\end{key}
\item  Do you think it is likely to have observed a mean like the one
  you got in \ref{xbar} if the arsenic concentrations in New Hampshire
  residents are not hazardous?
  
\begin{students}
  \vspace{1cm}
\end{students}
\begin{key}
   {\it Not sure.  Depends on how variable arsenic concentrations are!
}
\end{key}
\end{enumerate}


{\bf Step 4. Draw inferences beyond the data. }\\
We'll start with the first research question asked because we have
done confidence intervals for a single mean back in Activity 11. 

{\bf The First Research Question}: How high is the mean arsenic level
for New Hampshire residents with a private well? 
=======


\item  Large chain stores are always looking for locations into which
  they can expand -- perhaps into Bozeman. 
  When would a  decision to open a store in Bozeman  be wrong?\\
  When would a decision to not open a store in Bozeman  be
  wrong?\\
  Which is the worse error?  
\begin{students}
  \vspace{3cm}
\end{students}

\end{enumerate}

{\bf Two Types of Error. }

Definitions:\vspace{-.6cm}
\begin{itemize}
  \item To reject $H_0$ when it is true is called a Type I error.
  \item To fail to reject $H_0$ when it is false is called a Type II error.
\vspace{-.6cm}
\end{itemize}
To remember which is which: we start a hypothesis test by assuming
$H_0$ is true, so Type I goes with $H_0$ being true. 

This table also helps us stay organized: \hfill
\begin{tabular}{|l|c|c|}\hline
   & \multicolumn{2}{|c|}{Decision:} \\
$H_0$ is:  & Reject $H_0$ & Do not reject $H_0$\\\hline
true & {\em Type I Error} & Correct  \\ \hline
false& Correct & {\em Type II error}   \\ \hline\hline
\end{tabular}



{\bf Which is worse?}

The setup for hypothesis testing assumes that we really need to
control the rate of Type I error.  We can do this by setting our
 significance level, $\alpha$.  If, for example, $\alpha = 0.01$, then
 when we reject $H_0$ we are making an error less than 1\% of the time.
So $\alpha$ is the probability of making an error when $H_0$ is true.

There is also a symbol for the probability of a Type II error,
$\beta$, but it changes depending on which alternative parameter
value is correct. 
>>>>>>> f4001662d4281dfbcc417e2cc1ec4f6d088d24a6

\begin{enumerate}
 \setcounter{enumi}{8}
\item  Explain why this question is better answered using a confidence
  interval than by conducting a hypothesis test. 
\begin{students}
  \vspace{1cm}
\end{students}
\begin{key}
  {\it  Because we want to estimate the parameter (how high?), not
    compare it to some value.}
\end{key}

<<<<<<< HEAD
\item  Explain how you can use a deck of 19 cards to create the
  bootstrap distribution. (Go back to Activity 11 if you don't remember.)
\begin{students}
  \vspace{1cm}
\end{students}
\begin{key}
  {\it Write the original data on the cards, then sample with replacement 19
    times.}
\end{key}

% \item Write the original sample data on each card and write O (for
%   Original) on the upper right-hand corner.  Select one bootstrap
%   resample by shuffling the cards, selecting a card at random,
%   recording the value in the table below, returning the card to the
%   deck, and repeating 19 times. 

% \begin{students}
  
% \begin{center}
% \begin{tabular}{|r|r|r|r|r|r|r|r|r|r|} \hline
% \hspace*{1cm}& \hspace*{1cm}& \hspace*{1cm}& \hspace*{1cm}& \hspace*{1cm}& \hspace*{1cm}& \hspace*{1cm}& \hspace*{1cm}& \hspace*{1cm}& \hspace*{1cm}
% \\ \hline
% \hspace*{1cm}& \hspace*{1cm}& \hspace*{1cm}& \hspace*{1cm}& \hspace*{1cm}& \hspace*{1cm}& \hspace*{1cm}& \hspace*{1cm}& \hspace*{1cm} & \\ \hline
% \end{tabular}  
% \end{center}
% \end{students}

% What is the mean of your bootstrap  resample?  \vspace{1cm}

% \begin{students}
%   \vspace{1cm}
% \end{students}
% \begin{key}
%   {\it AWV}
% \end{key}

% \item  Combine your results with the rest of the class to create a
%   dotplot of the distribution for the average arsenic  level of New
%   Hampshire residents.   Draw the plot below. 
% \begin{students}
%   \vspace{4cm}
% \end{students}

% Is the plot centered where you expected?  Explain.
% \begin{students}
%   \vspace{1cm}
% \end{students}
% \begin{key}
%   {\it  Should be centered at the sample mean, 0.272 ppm. }
% \end{key}
\item Use the \fbox{One Quant} option in the web applet
  \url{https://jimrc.shinyapps.io/Sp-IntRoStats} to use the pre-loaded
  data (arsenic) and then generate a bootstrap distribution with 5000
  or more bootstrap statistics.  Draw the plot below and record the
  summary statistics.  % Show where your value from
 % the previous question falls on the plot. 
\begin{students}
  \vspace{4cm}
\end{students}
\begin{key}
  {\it \includegraphics[width=.5\linewidth]{../plots/arsenicCIplot.png}}
\end{key}

Explain how one dot on the plot was created and what it represents in
the context of the problem.
\begin{students}
  \vspace{1cm}
\end{students}
\begin{key}
  {\it One dot was created by sampling with replacement from the
    original data 19 times.  The mean level of arsenic concentration
    in the toenails of the resample is plotted.}
\end{key}

\item \label{SE2ci} Create a 95\% confidence interval using margin of
  error $ME =  2.11  \times SE$.
\begin{students}
  \vspace{1cm}
\end{students}
=======

\begin{center}
  {\bf Justice System and Errors }
\end{center}

Refer to this reading about the justice system:\\
\url{http://www.intuitor.com/statistics/T1T2Errors.html}  

In both the justice system and in statistics, we can make errors. In
statistics the only way to avoid making errors is to not state any
conclusion without measuring or polling the entire population.  That's
expensive and time consuming, so we instead try to control the chances
of making an error.  
 

For a scientist, committing a Type I error means we would report a
big discovery when in fact, nothing is going on. (How embarrassing!)
This is deemed more critical than a Type II error, which happens if
the scientist does a research project and finds no ``effect'' when, in
fact, there is one. 


Type II error is harder to control because it depends on these things:
\vspace{-.2cm}
\begin{itemize}
\item  The null hypothesis has to be wrong, but it could be wrong just
  by a small amount or by a large amount.  For example if 
   we did not reject the null hypothesis that treatment and
  control  were equally effective,  we could be making a type II
  error.  If in fact, if there was a small difference, it would be
  hard to detect, and if the treatment was far better, it would
  be easy to detect.  This is called the effect size, which is
  [difference between null model mean and an alternative mean] divided
  by standard error.
\item  Sample size.  P--values are strongly affected by sample
  size. With a big sample we can detect small differences.  With small
  samples, only coarse or obvious ones.
\item  Significance level.  The fence, usually called $\alpha$ (alpha), is
  usually set at .10, .05 or .01 with smaller values requiring
  stronger evidence before we reject the null hypothesis and thus
  lower probability of error.  
\end{itemize}
>>>>>>> f4001662d4281dfbcc417e2cc1ec4f6d088d24a6

\begin{key}
  {\it $ 0.272 \pm 2.11(0.053) = (0.16, 0.38)$ ppm}
\end{key}

<<<<<<< HEAD
\item  \label{percentile95} Create a 95\% confidence interval using
  the bootstrap   Percentile Method.
\begin{students}
  \vspace{1cm}
\end{students}

\begin{key}
  {\it $ (0.177, 0.383)$ ppm}
\end{key}

\item  How similar are the confidence intervals in \ref{percentile95}
  and \ref{SE2ci}?
\begin{students}
  \vspace{1cm}
\end{students}

\begin{key}
  {\it Pretty close, the percentile one is a bit higher on both ends
    than the 2.1SE interval.}
=======
Instead of limiting the probability of Type II error, researchers more
often speak of keeping the power as large as possible.  Power is one
minus the probability of Type II error.  Go to the Power Demo page:
\url{http://shiny.math.montana.edu/jimrc/IntroStatShinyApps} and click
\fbox{Power Demo} under \fbox{One Quant}.
\begin{students}
\vspace{.5cm}
\end{students}


\begin{enumerate}
  \setcounter{enumi}{4}
  \item  Set Sample size to 8, SD to 2, Alternative Mean to 2, and
    significance level to 0.01. What is the power?
\begin{students}
 \vspace{1cm} %% 
\end{students}

\begin{key}
  {\it 0.484 }
>>>>>>> f4001662d4281dfbcc417e2cc1ec4f6d088d24a6
\end{key}

  

<<<<<<< HEAD
\item \label{percentile90}Would you expect a 90\% confidence interval
  to be wider or narrower?  Explain, then give a 90\% (percentile)
  confidence interval.
\begin{students}
  \vspace{2cm}
\end{students}

\begin{key}
  {\it Narrower, we need to capture fewer dots, so we can move in.
    (0.191, 0.365)} 
\end{key}

\item  Interpret the 90\% confidence interval from \ref{percentile90}.
\begin{students}
  \vspace{2cm}
\end{students}

\begin{key}
  {\it We are 90\% confident the true average arsenic level in
    toenails for new Hampshire residents  with a private well is
    between 0.191 and 0.365 ppm. }
=======
    Increase sample size until you get power just bigger than 0.80.
    How large a sample is needed?
\begin{students}
\vspace{1cm} %% 
\end{students}

\begin{key}
  {\it 13 }
\end{key}


  \item  \label{ss8a}  Return to sample size 8. Adjust SD to get power just over
    0.80. Do you make it larger or smaller?  What value worked? 
\begin{students}
 \vspace{1cm} %% 
\end{students}

\begin{key}
  {\it 0.484  Smaller, down to 1.4}
>>>>>>> f4001662d4281dfbcc417e2cc1ec4f6d088d24a6
\end{key}
\end{enumerate}



<<<<<<< HEAD
{\bf The Second Research Question}: Is the mean arsenic level
for New Hampshire residents with a private well above the 0.15 threshold? 


  There are two possibilities for why the sample average
was 0.272.  List them here and label them as the null and alternative
hypotheses also write the null and alternative in notation.\\
$H_0:$ 
\begin{students}
  \vspace{1cm}
=======

    What is your effect size?
\begin{students}
\vspace{1cm} %% 
>>>>>>> f4001662d4281dfbcc417e2cc1ec4f6d088d24a6
\end{students}
\begin{key}
  {\it   $ \mu = 0.15$:  The water is, on average safe, so we've just
  observed an unusually high sample.}
\end{key}

$H_a:$
\begin{students}
  \vspace{1cm}
\end{students}
\begin{key}
<<<<<<< HEAD
  {\it  $ \mu > 0.15$  The water is, on average, dangerously high in arsenic.}
\end{key}

Is the alternative hypothesis right-tail, left-tail, or two-tail?
\begin{students}
  \vspace{1cm}
\end{students}

\begin{key}
  {\it Right tailed.}
\end{key}

We can simulate the behavior of arsenic concentrations in New
Hampshire ground water if we assume the null hypothesis which gives a
specific value for the mean.  The two key ideas when creating the
reference distribution are:
\begin{itemize}
\item 
 The resamples must be consistent with the null hypothesis.
\item 
 The resamples must be based on the sample data.
\end{itemize}

  We can use cards like we did for the CI above,
but we have to change the values so that they are consistent with the
null, $\mu = 0.15$.

\begin{enumerate}
\setcounter{enumi}{16}
\item How you could modify the sample data so as to force the null
  hypothesis to be true without changing the spread?  (Do not spend
  more than 2 minutes on this question.)
\begin{students}
  \vspace{2cm}
\end{students}

\begin{key}
  {\it  AWV, but if they have read/if you have lectured on this, they
    should know to shift the data to be centered on the null value,
    then resample.} 
\end{key}

\item Next we will simulate one repetition of the 19 toenail values
  collected by creating a deck of 19 cards to simulate what the data
  would look like {\bf if the null hypothesis} were true.
    \begin{enumerate}
    \item  What is the null value in this study?
\begin{students}
  \vspace{2cm}
\end{students}

\begin{key}
  {\it 0.15}
\end{key}
\item  \label{shift}How far is the sample mean from this null value?
\begin{students}
  \vspace{2cm}
\end{students}

\begin{key}
  {\it 0.272 – 0.15 = 0.122 above the mean}
=======
  {\it $2 - 0 = 2/1.4 = 1.429$}
\end{key}

  \item \label{ss8b}  Return to SD = 2.  Change Alternative Mean to get power just
    over 0.80.  Did you make it larger or smaller?  What value did you
    settle on?
\begin{students}
 \vspace{1cm} %% 
\end{students}

\begin{key}
  {\it Larger, 2.9 }
\end{key}


    What is your effect size?
\begin{students}
 \vspace{1cm} %% 
\end{students}

\begin{key}
  {\it $(2.9 - 0)/2 = 2.9/2 = 1.45$}
\end{key}

  \item   How do the effect sizes in \ref{ss8a} and \ref{ss8b} compare?  
\begin{students}
 \vspace{1cm} %% 
\end{students}

\begin{key}
  {\it Larger effect  size in the latter. }
>>>>>>> f4001662d4281dfbcc417e2cc1ec4f6d088d24a6
\end{key}
\item We need to shift the original data so that is it centered on the null
      value. 
      Subtract the value from (b) from each of the data numbers to
      get:

<<<<<<< HEAD
\begin{center}
\begin{tabular}{|r|r|r|r|r|r|r|r|r|r|} \hline
-0.003 &-0.004 &-0.023 &-0.004 &0.153 &0.236 &-0.042 &0.036 &0.188 &-0.017
\\ \hline
-0.049 &0.710 &0.395 &0.729 &0.147 &0.311 &0.019 &0.013 &0.053
& \\ \hline
\end{tabular}  
\end{center}
      What is the mean of the above values? Why do we want this
      to be the mean?
\begin{students}
  \vspace{1cm}
\end{students}

\begin{key}
  {\it 0.15, because this is the null value. }
\end{key}


\end{enumerate}
\item To speed up the process, we use  \fbox{ Test} option under
  \fbox{One Quant} at  \url{https://jimrc.shinyapps.io/Sp-IntRoStats
  }. You should have already loaded the data, but if not, go back to
  \fbox{Enter/Load Data} and select the preloaded \fbox{Arsenic}
  data. 

  \begin{itemize}
    \item Above the main plot, change the value for the null
      hypothesis to the one in our null.  (the just barely safe level)
      The software will shift the data to have this mean.
   \end{itemize}

    Look at the box on the right labeled {\sf Original Sample}.
       Does the mean match your answer to \ref{xbar}?  If not, consult
       with your instructor! 
\begin{students}
  \vspace{1cm}
\end{students}

\begin{key}
  {\it Look for data entry errors. }
\end{key}

\item   What is the statistic from the first  resample?
\begin{students}
  \vspace{1cm}
\end{students}

\begin{key}
  {\it AWV}
\end{key}

\item Where in the  output did you find it?
\begin{students}
  \vspace{1cm}
\end{students}

\begin{key}
  {\it Below center tables.}
\end{key}


\item Explain in the context of the problem what the one dot on the
  main plot represents. 
\begin{students}
  \vspace{1cm}
\end{students}

\begin{key}
  {\it The (resample) mean arsenic concentration in toe nails of 19
    New Hampshire residents who use a private well if the true arsenic
    concentration is 0.15 (right at the edge of being  hazardous) }
\end{key}
\item Generate 5000 or more randomization samples.  Copy the summary statistics and
the plot of the randomization distribution below
\begin{students}
  \vspace{4cm}
\end{students}

\begin{key}
  {\it \includegraphics[width=.5\linewidth]{../plots/arsenicNullDistn.png}}
\end{key}


\item 
 Where is the distribution centered?  Why does that make sense?
\begin{students}
  \vspace{1cm}
=======


    How do SD and Alternative Mean work together to determine power? 
\begin{students}
 \vspace{2cm}
\end{students}

\begin{key}
  {\it     Larger effect for same SD means more power, lower SD for same
    effect means more power.  In general, larger effect and lower
    SD means more power. }
\end{key}


  \item  Change significance level to 0.05.  What happens to power?
\begin{students}
 \vspace{1cm} 
\end{students}

\begin{key}
  {\it  .971}
\end{key}


    Change it to 0.10. What is the power? 
\begin{students}
 \vspace{1cm}%% 
\end{students}

\begin{key}
  {\it .992}
\end{key}

  \item    In which direction does power change when we decrease the
    significance level?  
\begin{students}
 \vspace{3cm}
\end{students}

\begin{key}
  {\it     It would decrease (power and significance level change in the
    same direction). }
\end{key}


  \item   Suppose that we are planning to do a study of how energy
    drinks effect RBAN scores similar to the study we read about in
    Activity 14.  From previous data, we have an estimate of standard
    deviation of 3.8. We plan to use a significance level of $\alpha =
    .05$, and want to be able to detect an increase in mean RBAN score
    of 2 with 90\% power.   How large must our sample size be?
\begin{students}
 \vspace{1cm} %%
\end{students}

\begin{key}
  {\it  33}
\end{key}


      If we choose $\alpha = .01$, how large a sample is needed?
\begin{students}
 \vspace{1cm}
>>>>>>> f4001662d4281dfbcc417e2cc1ec4f6d088d24a6
\end{students}
\begin{key}
<<<<<<< HEAD
  {\it Again, centered at 0.15, because this is the null value. }
\end{key}

Remember why we conducted this simulation: to assess whether the
observed result (mean of 0.272) would be unlikely to occur by chance
alone if the ground water in New Hampshire is not hazardous. 

\item Locate the observed result on the randomization distribution.
  Does it appear to be likely or unlikely to occur under the null
  hypothesis?  Explain your reasoning.
\begin{students}
  \vspace{1cm}
\end{students}

\begin{key}
  {\it Pretty fair in the right tail.  Appears unlikely to occur under
    the null hypothesis. } 
=======
\ \ \   {\it 50}
\end{key}

  \item  Now suppose that we are using a memory test used to study
    sleep deprivation.  Historical data 
    provides an estimate of SD = 13.  We want to use $\alpha = .05$ and
    need to detect an decrease in mean score (when people are sleep
    deprived) of 6 with 80\% power.  
    How large a sample is needed? 
\begin{students}
 \vspace{1cm} %%
\end{students}

\begin{key}
  {\it  31}
>>>>>>> f4001662d4281dfbcc417e2cc1ec4f6d088d24a6
\end{key}

\item \label{pval13}Just how unlikely is the observed result?  Calculate your
  p-value using the web app and the appropriate direction and cutoff value.
\begin{students}
  \vspace{1cm}
\end{students}

\begin{key}
  {\it Right tail p-value = 0.016}
\end{key}

<<<<<<< HEAD

 How many  resamples had a mean at least as extreme as the
 observed result?  
\begin{students}
  \vspace{1cm}
\end{students}

\begin{key}
  {\it AWV}
\end{key}

\item How strong is the evidence against the null hypothesis?  Look back to
  the guidelines for assessing strength of evidence using p-values
  given on page 35.
\begin{students}
  \vspace{1cm}
\end{students}

\begin{key}
  {\it p-value = 0.016, which is strong to very strong evidence
    against the null.}
=======
    If we want to limit the chance of Type II error to 10\% or less,
    how large a sample size is needed? 
\begin{students}
 \vspace{1cm}\\
\end{students}

\begin{key}
  {\it  Type II = 1--power so we want more than 90\% power.  Need 43
      people. }
\end{key}
\item Suppose we do another study on energy drinks with alcohol using
  Control and REDA.  This time we test hand-eye coordination using
  $H_0:\ \mu_{control} = \mu_{REDA}$ versus alternative  $H_a:\
  \mu_{control} > \mu_{REDA}$.
  \begin{enumerate}
  \item What would be a Type I error in this context? 
\begin{students}
 \vspace{1cm}\\
\end{students}

\begin{key}
  {\it   To conclude that there is a difference in mean coordination between the
    two treatments when, in fact, REDA has no effect on coordination scores. }
\end{key}
  \item What would be a Type II error in this context? 
\begin{students}
 \vspace{1cm}\\
\end{students}

\begin{key}
  {\it  To fail to find a difference in mean coordination between the
    two treatments when, in fact, REDA lowers coordination scores. }
>>>>>>> f4001662d4281dfbcc417e2cc1ec4f6d088d24a6
\end{key}
  \end{enumerate}

<<<<<<< HEAD


{\bf Step 5: Formulate conclusions.}


\item Based on this analysis, what is your conclusion about the
  residents in New Hampshire who own a private well based on this
  study?
\begin{students}
  \vspace{2cm}
\end{students}

\begin{key}
  {\it We have strong evidence that the true mean arsenic level in
    toenails of NH residents drinking from wells is greater than the
    hazardous threshold of 0.15 ppm.} 
\end{key}
\item Can you extend your results to all of New Hampshire residents?
  All New Hampshire residents with a private well?  Explain your
  reasoning.
\begin{students}
  \vspace{4cm}
\end{students}
=======
\end{enumerate}

>>>>>>> f4001662d4281dfbcc417e2cc1ec4f6d088d24a6

\begin{key}
  {\it Our evidence does not extend to all NH residents drinking from
    wells because we don't know that this is a representative sample.
   The data does not include people on municipal water systems, so it
   certainly does not extend to all NH residents.} 
\end{key}

\end{enumerate}

\begin{center}
<<<<<<< HEAD
  {\bf Take Home Messages}\vspace{-.1in}
=======
  {\bf Take Home Message} \vspace{-.6cm}
>>>>>>> f4001662d4281dfbcc417e2cc1ec4f6d088d24a6
\end{center}

\begin{itemize}
<<<<<<< HEAD
\item We first reviewed building a CI for a single mean.
\item You need to know when to discuss means versus proportions.  If
  the response has two categories, then we deal with proportions.  If
  the response is quantitative, then we estimate and test means.
\item The new twist today was to do a simulation for testing $H_0: \mu
  = \mu_o$ that the mean is some particular value.  We had to modify
  the data to make $H_0$ true, shifting it from its center at $\xb$ to
  be centered at $\mu_o$.  Then we resampled it as if for a bootstrap
  confidence interval, and located the observed statistic ($\xb$) to
  see how far out in the tails it was (the p--value).

 \item 
  Use the remaining space for any questions and your own summary of the
  lesson. \vfill

\end{itemize}



\noindent
{\bf Assignment} \vspace{-.2in}
\begin{itemize}
\item D2Quiz 7 is the next assignment.  Fill it in online.
 %%  We strongly encourage you to get help in the Math Learning Center.
\item Fill in the simulation confidence interval box in column 2 of
  the Review Table.
\item Read the next two pages before your next class.
\end{itemize}
=======
  \item Errors happen.  Use of statistics does not prevent all errors,
    but it does limit them to a level we can tolerate. We have labels
    for two types of error.
  \item The talk about probability of  error is based on the sampling
    distribution assuming random assignment of treatments or random
    sampling. It's really a ``best case'' scenario, because there
    could be other sources of error we have not considered.  For
    example, we could have not sampled from some part of the
    population, or we could have errors in our measuring tools.
  \item If you are designing a study, you might need to consult a
    statistician to help determine how large a sample size is
    needed. You'll need to decide what $\alpha$ to use, what the
    underlying variation is ($\sigma$), and how large a difference you
    need to detect with a certain level of power.
 \item 
  Use the remaining space for any questions or your own summary of the
  lesson. 
\end{itemize}
>>>>>>> f4001662d4281dfbcc417e2cc1ec4f6d088d24a6

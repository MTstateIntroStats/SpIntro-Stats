\def\theTopic{Reading 13}

\begin{center}
  {\bf Hypothesis Test for a Single Mean}
\end{center}

Watch the video: Hypothesis Test for a Single Mean found:\\
\url{https://camtasia.msu.montana.edu/Relay/Files/j63r458/Single_Mean_Hypo_Test/Single_Mean_Hypo_Test_-_20151018_222435_23.html}

{\bf Questions}:
\begin{itemize}
  \item Why might people care about average snow depth in the
    mountains around Bozeman and whether or not it has changed
    recently?\vfill
  \item Over the 30 years before 2011, what was average snow depth at
    Arch Falls?\vfill %% 38 in
  \item Over the past 5 years, what has average snow depth at Arch
    Falls been? \vfill%%34.7
  \item What variable is of interest? Is it quantitative or
    categorical? What statistic will we use to summarize it?\vfill
  \item What are the null and alternative hypotheses?\vfill
  \item Step 2 -- to create the null distribution -- uses a technique
    you've never seen before, and it's a bit weird. We'll work on it
    in the next class activity, so if you don't get it from the video,
    that's OK.\vfill
  \item Aside from getting the simulated distribution, everything else
    should seem familiar. Why do we want the distribution to have the
    center it does?\vfill
  \item How do we determine the p-value in this case? What is it?\vfill
  \item What do we conclude?\vspace*{\fill}
\end{itemize}



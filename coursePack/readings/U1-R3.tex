\def\theTopic{Reading 3}

\begin{center}
{\bf {\large Helper Hinderer}}
\end{center}


Nature 450, 557-559 (22 November 2007) | doi:10.1038/nature06288; Received 3 August 2007; Accepted 24 September 2007

Social evaluation by preverbal infants

J. Kiley Hamlin1, Karen Wynn1 \& Paul Bloom1

    Yale University, Department of Psychology, New Haven, Connecticut
    06520-8205, USA 


Abstract

The capacity to evaluate other people is essential for navigating the
social world. Humans must be able to assess the actions and intentions
of the people around them, and make accurate decisions about who is
friend and who is foe, who is an appropriate social partner and who is
not. Indeed, all social animals benefit from the capacity to identify
individual conspecifics that may help them, and to distinguish these
individuals from others that may harm them. Human adults evaluate
people rapidly and automatically on the basis of both behaviour and
physical features, but the ontogenetic origins and
development of this capacity are not well understood. Here we show
that 6- and 10-month-old infants take into account an individual's
actions towards others in evaluating that individual as appealing or
aversive: infants prefer an individual who helps another to one who
hinders another, prefer a helping individual to a neutral individual,
and prefer a neutral individual to a hindering individual. These
findings constitute evidence that preverbal infants assess individuals
on the basis of their behaviour towards others. This capacity may
serve as the foundation for moral thought and action, and its early
developmental emergence supports the view that social evaluation is a
biological adaptation. 

The following were counterbalanced across subjects in each age group:
(1) colour/shape of helper and hinderer; (2) order of helping and
hindering habituation events; (3) order of choice and looking time
measures; (4) positions of helper and hinderer in choice and in
looking time trials; and (5) order of 'approach-helper' and
'approach-hinderer' looking time trials. 

 \begin{center}
   {\large\bf Important Points}
 \end{center}
 \begin{itemize}
 \item Randomness
 \item What is the research question?
 \item What response was recorded?
 \end{itemize}

More on data summary?




\def\theTopic{Confidence }
\def\dayNum{7}

\begin{center}
\vspace*{-.2in}
{\bf {\large What Does ``Confidence'' Mean?}}\\
\end{center}

We've seen that we can obtain several different interval estimates for
a parameter. \\
  Each is centered at the statistic (here $\phat$), but they have
  different widths.

To understand the meaning of the term ``confidence'', you have to step
back from the data at hand and look at the process we use to create
the interval.
\begin{enumerate}
  \item Select a random sample from a population, measure each unit,
    and compute a  statistic like $\phat$ from it.
  \item Resample based on the statistic to create the interval.
\end{enumerate}

 To check to see how well the techniques work, we have to take a
 special case where we actually know the true parameter value.
 Obviously, if we know the value, we don't need to estimate it, but we
 have another purpose in mind: we will use the true value to generate
 many samples, then use each sample to estimate the parameter, and
 finally, we can check to see how well the confidence interval
 procedure worked by looking at the proportion of intervals which
 succeed in capturing the parameter value we started with.

 Again go to \url{http://shiny.math.montana.edu/jimrc/IntroStatShinyApps/}
 and select ``Confidence Interval Demo'' from the ``One Categ'' menu.
 
 The first slider on this page allows us to set the sample size --
 like the number of rat pairs in the experiment.  Let's start with
 \fbox{40}.\\
 The second slider sets the true proportion of successes for each
 trial or spin (each rat pair).  Let's set that at \fbox{0.75} or 75\%
 which is close to the observed $\phat$ of the rat study.\\
 You can then choose the number of times to repeat the process (gather
 new data and build a confidence interval [CI]: 10, 100, 1000 or 10K
 times) and the level of confidence you want (80, 90, 95, or 99\%).\\
 We'll start wih \fbox{100} simulations of a \fbox{90}\% CI.

  The upper plot shows 100  $\phat$'s -- one from each of the 100 simulations.
  \\
  The second plot shows the interval estimate we get from each
  $\phat$.  These  are stacked up to put smallest estimates on the
  bottom, largest on top. The vertical axis has no real meaning. 

  \begin{enumerate}
    \setcounter{enumi}{13}
  \item   Click on a point in the first plot to see its corresponding CI in
  the second plot.  Especially try the largest and smallest points.
  Which intervals do they create (in terms of left or right position)?
\begin{students}
  \vspace{1cm}
\end{students}
\begin{key}
  {\it lowest and highest CI's, resp.}
\end{key}
\item There is a light gray vertical line in the center of the
  plot. What is the value (on the $x$ axis) for this plot and why is
  it marked?
\begin{students}
  \vspace{1cm}
\end{students}
\begin{key}
  {\it It is the true parameter value: 0.75 if you followed the directions.}
\end{key}
\item What color are the intervals which do not cross the vertical
  line? How many are there?
\begin{students}
  \vspace{1cm}
\end{students}

\begin{key}
  {\it red, about 10}
\end{key}

\item What color are the intervals which cross over the vertical
  line? How many are there?
\begin{students}
  \vspace{1cm}
\end{students}

\begin{key}
  {\it green, about 90}
\end{key}

\item Change the confidence level to \fbox{95}\%. Does the upper plot
  change?  Does the lower plot?  Describe any changes.
\begin{students}
  \vspace{1cm}
\end{students}
\begin{key}
  {\it Upper plot should not change. Each interval in the lower plot
    gets longer, so some that were red may turn green now.}
\end{key}


\item If you want an interval which is stronger for confidence
  (has a higher level), what will happen to its width?
\begin{students}
  \vspace{.6cm}
\end{students}
\begin{key}
  {\it it must be wider}
\end{key}

  \item Try each confidence level in turn and record the coverage rate
    (under plot 2) for each.\\
    \begin{tabular}{|r|r|r|r|} \hline
      {\Large 80} &  {\Large 90} &  {\Large 95} &  {\Large 99}\\ \hline
   {\Large  \phantom{90} } & {\Large \phantom{90}  } &  {\Large
     \phantom{90} } &  {\Large  \phantom{90} } \\
       & & & \\ \hline
    \end{tabular}

  \end{enumerate}


\begin{center}
  {\large \bf Take Home Message} 
\end{center}

\begin{itemize}
\item Interval estimates are better than point estimates.
\item Our confidence in a particular interval is actually in the
  process used to create the interval.  We know that using this
  process over and over again (go out and collect a new random sample
  for each time) gives intervals which will usually
  cover the true value.\\
   We cannot know if a particular interval covered or not, so you have
   to be tolerant of some uncertainty.
 \item 
  Use the remaining space for any questions or your own summary of the
  lesson. 
\end{itemize}




